

% Errata list, if you have errors in the publications.
\errata

%
% icml 2014
\addpublication[conference]{Hongyu Su, Aristides Gionis, Juho Rousu}{Structured Prediction of Network Response}{Proceedings of the $31^{th}$ International Conference on Machine Learning (ICML 2014)}{Beijing, China, 2014. JMLR W{\&}CP volume 32:442-450}{June}{2014}{Copyright 2014 by the authors}{su14a}
\addcontribution{
\citepub{su14a} presents a novel formalism of the network response prediction problem, and proposes a structured output prediction algorithm for the problem.
The algorithm captures the contextual information and achieve a superior performance compared to several the state-of-the-art models.

The initial modeling idea was developed jointly by the authors.
The problem was jointly formulated.
The optimization algorithm was mainly contributed by the author and Prof. Rousu.
Prof. Gionis had major contribution to the idea of \sdp\ inference.
The author implemented the algorithm, designed and conducted the experiment, and analyzed the results.
All author participated in designing and writing the article.
}
%\adderrata{This is wrong}
\addpublicationpdf{publications/su14a.pdf}



%
% prib2010
\addpublication[conference]{Hongyu Su, Markus Heinonen, Juho Rousu}{Multilabel Classification of Drug-like Molecules via Max-margin Conditional Random Fields}{Proceedings of the $5^{th}$ International Conference on Pattern Recognition in Bioinformatics (PRIB 2010)}{Nijmegen, The Netherlands, 2010. Springer LNBI volume 6282:265-273}{September}{2010}{Copyright 2014 by the authors}{su10}
\addcontribution{
The novelty of \citepub{su10} is to develop a structured output prediction model and apply it on drug-bioactivity prediction problem.
The advanced model can captures the interdependency between cancer cell line targets.
Thus the model enjoys better prediction performance compared to previous single-task prediction models.

The original modeling idea was jointly developed.
The author contribute to the designing of the experiments and executing the algorithm.
All author participated in writing the articles.
}
%\adderrata{This is wrong}
\addpublicationpdf{publications/su10.pdf}


%
% prib 2011
\addpublication[conference]{Hongyu Su, Juho Rousu}{Multi-task Drug Bioactivity Classification with Graph Labeling Ensembles}{Proceedings of the $6^{th}$ International Conference on Pattern Recognition in Bioinformatics (PRIB 2011)}{Delft, The Netherlands, 2011. Springer LNBI volume 7035:157-167}{November}{2011}{Copyright 2014 by the authors}{su11}
\addcontribution{
\citepub{su11} presents an ensemble approach for structured output prediction problem with application in molecular activity prediction.
The model extends \citepub{su10} by tackling the situation where the structure of the output variables is not known, which largely widens the applicability of the structured output prediction models.

The idea of the ensemble learning strategy was jointly developed. 
In addition, the author contributed mainly to the algorithm design and implementation.
The author performed the experiments and analyzed the results.
The article was structured by authors and was written jointly.
}
%\adderrata{This is wrong}
\addpublicationpdf{publications/su11.pdf}

%
% mlj 2014
\addpublication{Hongyu Su, Juho Rousu}{Multilabel Classification through Random Graph Ensembles}{Machine Learning}{Volume, issue, pages, and other detailed information}{Month}{Year}{Copyright 2014 by the authors}{su14b}
\addcontribution{
\citepub{su14b} generalizes \citepub{su11} in several ways.
By extending \citepub{su11}, the article introduces two other elegant aggregation frameworks for multi-task structured output prediction.
A theory is also developed in this article  which explains the performance improvement of the proposed model.
It also extends the field of application from only molecular activity prediction problem to a set of heterogeneous multi-task prediction problems.

The original idea of generating aggregation strategy for structured output prediction was developed jointly by authors.
The author designed and implemented the algorithms that bring the idea into practice.
The theory part of the article is mainly contributed by the author.
The article was structured by authors and was written jointly.
}
%\adderrata{This is wrong}
\addpublicationpdf{publications/su14b.pdf}




%
% nips 2014
\addpublication[conference]{Mario Marchand, Hongyu Su, Emilie Morvant, Juho Rousu, John Shawe-Taylor}{Multilabel Structured Output Learning with Random Spanning Trees of Max-Margin Markov Networks}{Proceedings of the $28^{th}$ Advances in Neural Information Processing Systems (NIPS 2014)}{to appear}{December}{2014}{Copyright 2014 by the authors}{su14c}
\addcontribution{
\citepub{su14c} is a major step forward of \citepub{su14b} by introducing the joint learning and inference framework and developing rigours learning theory to backup the algorithm.

This article is the main outcome of author's research visit to University College, London.
The idea was initialzed jointly by the authors prior to the visit.
The theory is mostly contributed by Prof. Marchand, where the author also joint the discussion and the development.
The author contribute mainly to the designing and the implementation of the learning and optimization framework.
The author conducted the experiments and analyze the results.
All authors participated in writing the article.
}
%\adderrata{This is wrong}
\addpublicationpdf{publications/su14c.pdf}


\iffalse
%
% mlj 2014
\addpublication{Hongyu Su, Juho Rousu}{Multilabel Classification through Random Graph Ensembles}{Machine Learning}{Volume, issue, pages, and other detailed information}{Month}{Year}{Copyright 2014 by the authors}{su14b}
\addcontribution{
\citepub{su14b} extends \citepub{su13} by adding extensive literature review, details in developing the algorithm, and more experimental results.
It also links the model performance to the existing theory by providing empirical evidence.

The article was designed and written jointly.
The experiments are mostly contributed by the author.
}
%\adderrata{This is wrong}
\addpublicationpdf{publications/su14b.pdf}

%
% acml 2014
\addpublication[conference]{Hongyu Su, Juho Rousu}{Multilabel Classification through Random Graph Ensembles}{Proceedings of the $5^{th}$ Asian Conference on Machine Learning (ACML 2013)}{Canberra, Australia, 2013. JMLR W{\&}CP volume 29:404-418}{November}{2013}{Copyright 2013 by the authors}{su13}
\addcontribution{
\citepub{su13} generalizes \citepub{su11} in several ways.
The article introduces two advanced ensemble frameworks for multilabel structured output prediction, and describes a theory to explain the advantage of the proposed models.
It also extends the application from drug-bioactivity prediction problem to other applications.

The original idea of generating ensembles was developed jointly by authors.
The author developed the algorithms that bring the idea into effect
The theory was mainly developed by the author.
The article was written jointly.
}
%\adderrata{This is wrong}
\addpublicationpdf{publications/su13.pdf}
\fi


